\documentclass[a4paper, 12pt]{article}
\usepackage[T1]{fontenc}
\renewcommand{\familydefault}{\sfdefault}
% ! TEX TS-program = xelatex
% ! TEX encoding = UTF-8 Unicode

%%========== Main packages ========== %%
\usepackage{amsmath}
\usepackage{graphicx}
\usepackage{amssymb}
\usepackage{listings}
\usepackage{float}
\usepackage{color}
\usepackage{xcolor}

%%========== Colors ===========%%
\definecolor{deepblue}{rgb}{0,0,0.5}
\definecolor{deepred}{rgb}{0.6,0,0}
\definecolor{deepgreen}{rgb}{0,0.5,0}
\definecolor{codegreen}{rgb}{0,0.6,0}
\definecolor{backcolour}{rgb}{0.95,0.95,0.95}

%%========== Beamer packages and Settings ========== %%
%% \usepackage{bm}
%% \setbeamercolor{footnote mark}{fg=red}
%% \usetheme{CambridgeUS}
%% \setbeamerfont{footnote}{size=\tiny}
%% \setbeamertemplate{footline}[frame number]

%%========== Listings Settings ========== %%
\newcommand\pythonstyle
{\lstset
    {
        frame=tb,
        language=Python,
        aboveskip=5mm,
        belowskip=5mm,
        backgroundcolor=\color{backcolour},
        morekeywords={self},
        keywordstyle=\bfseries\color{deepblue},
        identifierstyle=\ttfamily\color{blue},
        emph={class,__init__,def},
        emphstyle=\ttfamily\color{deepred},
        stringstyle=\color{deepgreen},
        commentstyle=\itshape\color{codegreen},
        basicstyle=\ttfamily,
        breakatwhitespace=true,
        breaklines=true,
        keepspaces=true,
        numbers=none,
        showspaces=false,
        showstringspaces=false,
        showtabs=false,
        tabsize=4,
    }
}

\newcommand\cppstyle
{\lstset
    {
        language=C++,
        backgroundcolor=\color{backcolour},
        morekeywords={include, using, namespace, std, cout, cin, endl},
        keywordstyle=\ttfamily\color{deepblue},
        emph={class, public, private, protected, int, float, double, char, string, bool, void, return, if, else, for, while, do, switch, case, break, continue, default, true, false},
        emphstyle=\ttfamily\color{deepred},
        stringstyle=\color{deepgreen},
        commentstyle=\color{codegreen},
        basicstyle=\ttfamily\footnotesize,
        breakatwhitespace=false,
        breaklines=true,
        keepspaces=true,
        numbers=none,
        showspaces=false,
        showstringspaces=false,
        showtabs=false,
        tabsize=4,
    }
}

\newcommand\bashstyle
{\lstset
    {
        frame=tb,
        language=bash,
        backgroundcolor=\color{backcolour},
        morekeywords={include, using, namespace, std, cout, cin, endl},
        keywordstyle=\ttfamily\color{deepblue},
        emph={class, public, private, protected, int, float, double, char, string, bool, void, return, if, else, for, while, do, switch, case, break, continue, default, true, false},
        emphstyle=\ttfamily\color{deepred},
        stringstyle=\color{deepgreen},
        commentstyle=\color{codegreen},
        basicstyle=\ttfamily\footnotesize,
        breakatwhitespace=false,
        breaklines=true,
        keepspaces=true,
        numbers=none,
        showspaces=false,
        showstringspaces=false,
        showtabs=false,
        tabsize=4,
    }
}

\lstnewenvironment{python}[1][]
    {
    \pythonstyle
    \lstset{#1}
    }
{}

\lstnewenvironment{bash}[1][]
    {
    \bashstyle
    \lstset{#1}
    }
{}

\lstnewenvironment{cpp}[1][]
{
\cppstyle
\lstset{#1}
}
{}

\newcommand\pythoninline[1]{{\pythonstyle\lstinline!#1!}}
\newcommand\cppinline[1]{{\cppstyle\lstinline!#1!}}
\newcommand\bashinline[1]{{\bashstyle\lstinline!#1!}}
%% ========== Geometry Settings ==========
\usepackage{fancyhdr}
\usepackage{geometry}
\geometry{top = 2cm, bottom = 2cm, right = 2cm, left = 2cm}
\usepackage{appendix}
\usepackage{setspace}

%% ========== Figure Settings ==========
\usepackage{graphicx}
\usepackage{float}
\usepackage{subcaption}

%% ========== Table Settings ==========
\usepackage{multirow}
\usepackage{booktabs}
\usepackage{longtable}
\usepackage{dcolumn}
\usepackage{ctable}
%% ========== Math Settings ==========
\usepackage{amsmath, mathtools, amssymb, empheq, amsthm, cancel, mleftright} % 
\DeclareMathOperator{\erf}{erf}
\DeclareMathOperator{\arcsec}{arcsec}
\DeclareMathOperator{\arccot}{arccot}
\DeclareMathOperator{\arccsc}{arccsc}
\DeclareMathOperator{\supp}{supp}
\DeclareMathOperator{\diag}{diag}
\DeclareMathOperator*{\argmax}{arg\,max}
\DeclareMathOperator*{\argmin}{arg\,min}
\renewcommand\qedsymbol{$\blacksquare$}
\allowdisplaybreaks
\newcommand{\ddx}{\frac{d}{dx}}
\newcommand{\dfdx}{\frac{df}{dx}}
\newcommand{\ddxp}[1]{\frac{d}{dx}\left(#1 \right)}
\newcommand{\dydx}{\frac{dy}{dx}}
\let\ds\displaystyle
\newcommand{\intx}[1]{\int #1 \, dx}
\newcommand{\intt}[1]{\int #1 \, dt}
\newcommand{\imp}{\Rightarrow}
\newcommand{\un}{\cup}
\newcommand{\inter}{\cap}
\newcommand{\ps}{\mathscr{P}}
\newcommand{\set}[1]{\left\{ #1 \right\}}

%% ========== TIKZ Settings ==========
\usepackage{tikz}
\usepackage{standalone}
\tikzset{>=latex} % for LaTeX arrow head
\usepackage{pgfplots} % for the axis environment
\usepackage{xcolor}
\usepackage[outline]{contour} % halo around text
\contourlength{1.2pt}
\usetikzlibrary{positioning,calc}
\usetikzlibrary{backgrounds}% required for 'inner frame sep'
\pgfplotsset{compat = 1.18}

%% ========== Reference Settings ==========
\usepackage{hyperref} % Set some hyperref style
\hypersetup{
    colorlinks = true,
    linkcolor = blue,
    filecolor = magenta,      
    urlcolor = sinopia,
    citecolor = blue,
}
\usepackage{natbib}
\usepackage{nccbbb}
\urlstyle{same}

%% ========== Algorithm ============
\usepackage[ruled]{algorithm2e}

%% ========== Font Settings ==========
\sffamily % sans serif font

\setlength{\headheight}{27.41830pt}
\sffamily
\definecolor{sinopia}{rgb}{0.8, 0.25, 0.33}
\pagestyle{fancy}
\fancyhf{}
\chead{
    TMBA Algorithmic Trading Summer Track HW1 \\
    25th Mentor, Yu-Chen (Abner) Den, 2024 Summer
}
\begin{document}
\section*{\center{Introduction}}
Welcome to Algorithmic Trading Club @ TMBA! This is your first
homework assignment. The purpose of this assignment is to get you
familiar with the basics of Python for financial data analysis and
constructing trading strategies. You don't need to write any report / slides
after finishing this assignment, it's just for you to get familiar with how to
build a backtest system for single-asset CTA trading strategies.

\subsection*{\center{Instructions}}

In the folder \pythoninline{src/}, you can see four Python files:

\begin{itemize}
    \item \pythoninline{main.py}
    \item \pythoninline{utils.py}
    \item \pythoninline{backtest.py}
    \item \pythoninline{__init__.py}
\end{itemize}

\noindent Where \pythoninline{main.py} is the main file for you to execute the
backtest system. You can run the backtest system by executing the following
command in the terminal:

\begin{python}
~/HWs> python -m src.main
\end{python}

\noindent In \pythoninline{utils.py}, you can find alot of useful functions
for you to use in the backtest system, such as \pythoninline{load_data()} for
loading the data, \pythoninline{plot_dd()} for plotting the drawdown of your strategy, etc. In
\pythoninline{backtest.py}, you can find functions for you to implement trading
indicators and trading strategies.

\subsection*{\center{Requirements}}

To start programming, you need to install the following packages if you haven't
done so. I have wrote them into the \pythoninline{requirements.txt} file, you can
just type the following command in the terminal to install them. A good habit is to
create a virtual environment before installing the packages to prevent packages version conflicts.

\begin{python}
~/HWs> python3 -m venv your_venv_name
~/HWs> source your_venv_name/bin/activate

(your_venv_name) ~/HWs> pip3 install -r requirements.txt
\end{python}

\noindent After these steps, you are ready to start programming!

\section*{\center{Tasks}}

I have already implemented a simple moving average crossover
strategy in \pythoninline{backtest.py}. You have two choices to
complete this assignment:

\begin{itemize}
    \item \textbf{Not familiar with Python and CLI \footnote{Command Line Interface}:}
        You can first understand the code and the backtest structure, and then try
        to run the code in your local environment successfully.
    \item \textbf{Familiar with Python and CLI:}
        You can try to implement a ``Moving Average Crossover'' strategy based on
        the current structure (Or other trading strategies if you want to do so).
\end{itemize}

\section*{\center{Hints}}

\subsection*{\center{Moving Average Crossover}}

To implement the moving average crossover strategy, you need to
first calculate one more moving average line, be aware that two
moving average lines need to have different periods (different \pythoninline{k}).

\begin{python}
get_moving_average(data, args.k1)
get_moving_average(data, args.k2)
\end{python}

\noindent To generate the trading signals, you can refer to the following links:
\begin{itemize}
    \item \url{https://trendspider.com/learning-center/moving-average-crossover-strategies/}
    \item \url{https://www.oanda.com/bvi-ft/lab-education/dictionary/golden-cross_death-cross/}
\end{itemize}

\subsection*{\center{CLI Execution}}

Type the following command in the terminal to run the backtest system:

\begin{python}
~/HWs> python -m src.main --k1 10 --k2 20
\end{python}

\subsection*{\center{Supplementary Materials -- \pythoninline{ArgumentParser}}}
\pythoninline{ArgumentParser} is a class in the \pythoninline{argparse} module
that allows you to parse command-line arguments. You can use it to add arguments
to your script, and then parse them. Here is an example of how to use it:

\begin{python}
"""
-*- encoding: utf-8 -*-
main.py

Make sure to first `pip install argparse`
"""
from argparse import ArgumentParser

def parse_args():
    parser = ArgumentParser()
    parser.add_argument("--k", type=int)
    return parser.parse_args()

if __name__ == "__main__":
    args = parse_args()
    print(args.k)
\end{python}

And you can run the script with the following command:

\begin{python}
~/HWs> python main.py --k 10
\end{python}

it will print out \pythoninline{10}.

\section*{\center{Want to know more?}}

\begin{itemize}
    \item The source code of this homework \& the full version of the
    backtest system can be found in my \href[]{https://github.com/AbnerTeng/Quant-Finance/tree/master}{GitHub repo -- Quant-Finance}
    \item \href[]{https://abnerteng.notion.site/Coding-Style-Manual-bab7b862daec4057bd6d3e5b68aa3721?pvs=4}{Coding Style Manual} wrote by myself.
    A good coding style can make you easier to find a job in quantitative trading industry / tech companies.
    \item \href[]{https://abnerteng.github.io/post/kronos_interview/}{Interview experience of Kronos Summer Analyst}. If you're interested in
    what the interview process of a quant trading firm looks like, you can check this out.
\end{itemize}

\noindent Any questions, feel free to ask me on Discord / Facebook. 

\end{document}
