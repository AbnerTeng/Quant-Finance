% ! TEX TS-program = xelatex
% ! TEX encoding = UTF-8 Unicode

%%========== Main packages ========== %%
\usepackage{amsmath}
\usepackage{graphicx}
\usepackage{amssymb}
\usepackage{listings}
\usepackage{float}
\usepackage{color}
\usepackage{xcolor}

%%========== Colors ===========%%
\definecolor{deepblue}{rgb}{0,0,0.5}
\definecolor{deepred}{rgb}{0.6,0,0}
\definecolor{deepgreen}{rgb}{0,0.5,0}
\definecolor{codegreen}{rgb}{0,0.6,0}
\definecolor{backcolour}{rgb}{0.95,0.95,0.95}

%%========== Beamer packages and Settings ========== %%
%% \usepackage{bm}
%% \setbeamercolor{footnote mark}{fg=red}
%% \usetheme{CambridgeUS}
%% \setbeamerfont{footnote}{size=\tiny}
%% \setbeamertemplate{footline}[frame number]

%%========== Listings Settings ========== %%
\newcommand\pythonstyle
{\lstset
    {
        frame=tb,
        language=Python,
        aboveskip=5mm,
        belowskip=5mm,
        backgroundcolor=\color{backcolour},
        morekeywords={self},
        keywordstyle=\bfseries\color{deepblue},
        identifierstyle=\ttfamily\color{blue},
        emph={class,__init__,def},
        emphstyle=\ttfamily\color{deepred},
        stringstyle=\color{deepgreen},
        commentstyle=\itshape\color{codegreen},
        basicstyle=\ttfamily,
        breakatwhitespace=true,
        breaklines=true,
        keepspaces=true,
        numbers=none,
        showspaces=false,
        showstringspaces=false,
        showtabs=false,
        tabsize=4,
    }
}

\newcommand\cppstyle
{\lstset
    {
        language=C++,
        backgroundcolor=\color{backcolour},
        morekeywords={include, using, namespace, std, cout, cin, endl},
        keywordstyle=\ttfamily\color{deepblue},
        emph={class, public, private, protected, int, float, double, char, string, bool, void, return, if, else, for, while, do, switch, case, break, continue, default, true, false},
        emphstyle=\ttfamily\color{deepred},
        stringstyle=\color{deepgreen},
        commentstyle=\color{codegreen},
        basicstyle=\ttfamily\footnotesize,
        breakatwhitespace=false,
        breaklines=true,
        keepspaces=true,
        numbers=none,
        showspaces=false,
        showstringspaces=false,
        showtabs=false,
        tabsize=4,
    }
}

\newcommand\bashstyle
{\lstset
    {
        frame=tb,
        language=bash,
        backgroundcolor=\color{backcolour},
        morekeywords={include, using, namespace, std, cout, cin, endl},
        keywordstyle=\ttfamily\color{deepblue},
        emph={class, public, private, protected, int, float, double, char, string, bool, void, return, if, else, for, while, do, switch, case, break, continue, default, true, false},
        emphstyle=\ttfamily\color{deepred},
        stringstyle=\color{deepgreen},
        commentstyle=\color{codegreen},
        basicstyle=\ttfamily\footnotesize,
        breakatwhitespace=false,
        breaklines=true,
        keepspaces=true,
        numbers=none,
        showspaces=false,
        showstringspaces=false,
        showtabs=false,
        tabsize=4,
    }
}

\lstnewenvironment{python}[1][]
    {
    \pythonstyle
    \lstset{#1}
    }
{}

\lstnewenvironment{bash}[1][]
    {
    \bashstyle
    \lstset{#1}
    }
{}

\lstnewenvironment{cpp}[1][]
{
\cppstyle
\lstset{#1}
}
{}

\newcommand\pythoninline[1]{{\pythonstyle\lstinline!#1!}}
\newcommand\cppinline[1]{{\cppstyle\lstinline!#1!}}
\newcommand\bashinline[1]{{\bashstyle\lstinline!#1!}}